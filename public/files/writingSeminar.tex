\documentclass[a4paper,11pt]{article}

\usepackage{amsmath,amsthm}
\usepackage[margin=2.5cm]{geometry}
\usepackage{ae,aecompl}
\usepackage[utf8]{inputenc}
\usepackage{textcomp}
\usepackage[T1]{fontenc}
\usepackage{hyperref}
\usepackage{setspace}
\onehalfspacing

\title{Some thoughts on writing a literature based seminar paper}
\author{Christoph Schottmüller}

\begin{document}
\maketitle

\begin{abstract}
This document gives some guidelines for  students who have to write a seminar paper based on an article in the literature.

\textbf{keywords:} academic writing, seminar papers
\end{abstract}

This document assumes that the reader is a student in the following situation. The student takes a course that in the German academic tradition is labeled as ``Seminar'' and in which the student has to write a paper based on an article in the literature.\footnote{Typically, the student also has to present his work to fellow students. However, the focus of this document is on the written paper.} The task consists of summarizing the assigned article and to produce some original contribution based on the paper. Both tasks are described in the following two sections. Further sections, discuss specific topics and questions that occur in the context of seminar papers.

One caveat in advance: I will assume that the assigned article is typical for the literature in (micro-)economic theory or theoretical industrial organization. Some of the points below apply also to empirical articles and articles of other disciplines while others do not.

\section{Summarizing the Article}
\label{sec:summarizing-article}

Most articles simplify this task by already providing summaries of various kinds. First, the abstract which is a summary in 100--200 words. Second, the introduction in which most authors describe their main contribution/mechanism/argument. Last but not least, many (but not all articles) reiterate their main points in the conclusion. It is, however, important to point out that the task for a seminar paper writing student is not to copy and paste one of these summaries. To start with, the summary in the seminar paper is normally longer than any of the summary elements of the article.

The summary in the seminar paper should (i) explain the main model setup of the paper, (ii) present the main results and (iii) give intuitive explanations for the main results. I discuss these three tasks in turn.

The main source for explaining the model setup is, of course, the ``Model'' section in the article. However, it may not be optimal to simply paraphrase this section. Ideally, you present only those parts that are necessary for understanding the rest of your seminar paper. It may, therefore, make sense to initially only draft this part of the seminar paper and to write it properly only when knowing the precise content of the rest of the seminar paper. Sometimes the use of graphical elements can help to explain the model setup in a concise manner. For example, a time line is often useful if the model in the article is dynamic. Note that it is absolutely fine to present a simplified version of the model in the paper as long as this simplified model is sufficient to obtain and explain the main results of the article.

The main results of the article are usually easy to find. They are formally stated as ``Theorems'' or ``Propositions'' and are typically mentioned in the abstract and the introduction. ``Lemmas'' on the other hand are usually of minor importance and often intermediate steps necessary to obtain the main results. Occasionally, however,  the key new step that allows to obtain the main result may be hidden in a lemma. In this case, the lemma has to be presented in the seminar paper. Note that formal results, like Theorems and Propositions, are often stated in very technical/mathematical language but the paragraph above or below the result will often contain a less formal and easier to read explanation.

These explanations can form the basis for the respective part of the seminar paper. In most cases, replicating the proof of a result from the article is an unnecessary waste of space (and also does not demonstrate mastery of the subject). What students are therefore asked is to give the intuition behind the (proof of the) result. In other words, the student has to explain why this result holds in this model; how it follows from the model assumptions, e.g. from the incentives of the players and their strategic interaction. Explaining the economic logic and mechanism behind the results is what sets economists and mathematicians apart.

\section{Original Contribution}
\label{sec:orig-contr}

While explaining the economic logic behind results in the literature already demonstrates a significant amount of economic insight, the main opportunity for showcasing one's abilities as a creative economic thinker is the original contribution. As originality is key, there exists no simple recipe for this part of the seminar paper. However, it is useful to know some typical ways of proceeding and I will provide a non-exhaustive list below.

Before doing so, it is useful to discuss what the purpose of the own contribution is (apart from showcasing abilities). The main idea is that the student conducts a little bit of research. Hence, the contribution is judged on the research value it delivers. Research value is provided by novelty, economic relevance and non-obviousness. In other words, the own contribution should deliver insights that are new and relevant. In the setting of a seminar paper, these insights are not expected to be gigantic but some new insight should exist. To clarify this, consider the possibility of developing a variation of the model (this is described in more detail below). One possibility is to view the variation as a robustness check for the published article. In this case, the most important question is whether the results of the variation are different or in line with results of the article. This robustness check is a valid contribution to economic research (as long as it is not totally obvious). It is even better if there is a good economic motivation for the variation, e.g. some real life example is given where the added feature matters. The results of the variation are then interpreted in light of this example.

This leads to another important point regarding the objective of the own contribution. While I will not describe this in the list of possible ways of how to create an own contribution below in order to avoid repetition, one should keep in mind that the objective is not simply to solve the equations, make the simulation etc., but to interpret the obtained results as an economist. Put differently, there should be a motivation for the own contribution and towards the end of the own contribution it should be clarified how the analysis relates to this motivation. An economic interpretation and/or discussion of the analytical or numerical results is absolutely mandatory. For now, however, I want to turn to the most common ways of producing an own contribution.\footnote{Not all of these possibilities are ``created equal''. Some tend to require more creativity than others and some tend to lead to more novel or more relevant results than others. Nevertheless, I believe that every grade is possible with each type of contribution.}

\paragraph{A variation of the model in the article.} This usually takes the form of changing one assumption. Some examples could be: considering a duopoly instead of a monopoly, introducing a public singal in addition to private signals, changing from Cournot to Hotelling competition, rearranging the timing of the game, adding another stage to the model (before or afterwards), using a triangular instead of a uniform distribution for some random variable etc. In all these cases, the model is likely to become more complicated than the one in the original article. Complicating matters is, generally speaking, not a good idea for a seminar paper: after all the authors of the article had -- compared to the student -- more training as economists and more time to conduct their research and they decided not to address the more complicated setup. However, it is often possible to simplify the model in another dimension first. For example, by assuming a specific cost function $c(x)=x^2$ instead of ``$c(x)$ be increasing and convex'' or a uniform distribution on $[0,1]$ instead of a ``$\phi$ be a continuous density defined on a compact interval'' etc. In other words, before extending a model you may want to simplify it. In this case, it is often helpful to check how the general results from the article sharpen under these simplifying assumption in order to have the right comparison point for the results of the variation.

\paragraph{A numerical solution/simulation that illustrates aspects of the model in the article that have not been or cannot be shown analytically.} In many articles, analytical solutions are not provided. That is, the equilibrium may be characterized by a (set of) equations or first order conditions but these equations cannot be solved for the equilibrium actions because either they are complicated expressions or because they contain general functions -- like $c(x,\theta)$. (The author of the article may still be able to show certain properties that hold in equilibrium, e.g. show how equilibrium choices change in the parameters of the model.) One possibility is then to use a computer program to solve the equations for various parameter values and to graphically illustrate this solution. Again it will be necessary to choose particular functional forms, e.g. $c(x,\theta)=x^2+\theta x$, to do so. It is particularly interesting to check whether additional properties of the solution, i.e. results that could not be obtained in the paper, appear in this numerical solution. For example, the author may have a general inverse demand function $D(p)$ and may not say anything about changes in demand but if this function is specified as $D(p)=A-p$ it is possible to compare the solution for different levels of $A$. This allows to analyze how the solution changes if demand increases (in the sense that $A$ increases).

In many ways, this kind of own contribution is similar to the variation of the model mentioned above. The only difference is that one does not extend but sticks to the model of the article with possibly additional simplification that make a numerical analysis possible. Clearly, students need a little knowledge of programming or computer algebra systems. To facilitate this kind of seminar papers, the chair provides some quick tutorials for how to numerically solve equations, maximize functions, compute expected values, create plots etc., see \url{https://schottmueller.github.io/#teaching}. Usually, the tools explained there are sufficient to carry out a seminar paper project. 

\paragraph{A worked-out example where you can illustrate the results of the paper (sometimes leading to a graphical illustration).} Similarly to the options discussed above, the idea is to assume concrete functional forms, e.g. $x^2$ instead of an increasing and convex function $c(x)$, and to provide an explicit solution, i.e. in equilibrium $x=\dots$. The difference to the numerical solution is that instead of using a computer, this solution is obtained by hand. Ideally, the solution still contains some parameter(s): for example instead of $c(x)$ one could choose $\alpha x^2$ where $\alpha>0$ is a parameter. This allows to conduct comparative statics with respect to $\alpha$, i.e. it is possible to analyze how the solution changes if (marginal) costs increase through an increase in $\alpha$. Obviously, this kind of own contribution applies only to articles in which an explicit solution is not provided in the article itself. As before, the most interesting question is whether additional results hold due to the more specific assumptions one makes. Explicit solutions might also allow to create graphical illustrations, e.g. how equilibrium choices depend on parameters.

\paragraph{A comparison of the results in the paper to later, related articles (if this is not made in those later articles).} The task consists first of finding related articles and then to clarify in what way these articles differ from the assigned articles. The hard bit, and that is the bit where you are providing an own contribution, is to explain how and why differences in model setups and assumptions translate into differences in the results. The student's task is to describe the mechanisms at work intuitively  so that a reader familiar with the assigned article (from the summary the student provides in the first part) understands why later articles find different results when they change the setup in the way they do.

\paragraph{Formulating and solving a game-theoretic model addressing points made only verbally in the assigned article.} This way of contribution tends to be particularly suited for old articles and articles published in interdisciplinary or non-economics journals. The reason is that these articles often rely more on verbal arguments than articles in the leading journals for economic theory. The task is self-explanatory: One clarifies the statement analyzed, sets up a game theoretic model that is in line with the verbal argument and solves this model. The main objective is to check whether the argument is really correct. In other words, the mathematical formalism of game theory is more transparent than verbal arguments and requires strictly logical operations. Sometimes flaws in the verbal reasoning can be discovered or it can be clarified which implicit assumptions the author makes. That is, the game theoretic analysis can show which assumptions are necessary to obtain the conclusion of the article.

\paragraph{An empirical check of the results of the article.} A theoretical article may make testable predictions and in this case testing those predictions constitutes a strong contribution. However, empirical research is usually not suitable for a seminar paper due to the limited time frame. In particular finding suitable data and bringing this data in reasonable shape for empirical tests is often infeasible. Students who are very familiar with programming may consider webscraping as a way of collecting a fitting data set. 

\paragraph{Designing an experiment.} Another way of testing implications of a model is to conduct a laboratory experiment. Conducting a proper lab experiment is expensive and time intensive and consequently beyond the scope of a seminar paper. However, setting up everything necessary in order to conduct the experiment constitutes already a contribution. In this case, the student explains the experiment including the different treatments, the specific parameter values used, determines through a power calculation the necessary number of subjects and formulates the hypotheses to be tested. Furthermore, the student sets up a pre-analysis plan describing how the experimental data will be analyzed in order to test the hypotheses. A student may -- if time permits -- even go further by programming the experiment. If consideration of non-standard preferences motivate the experiment, play of players with such non-standard preferences (or using heuristics) may be simulated and compared with equilibrium play by players with standard preferences.

\paragraph{Combining the models of two articles.} Combining models and insights from two distinct literatures is a common practice in academic research. Again this may prove to be too time intensive for a seminar paper unless the models are simplified in advance (see the paragraph on a ``variation of the model in the article''). The combination of two literatures ideally gives new insights into both topics/problems. 

\section{Structure}
\label{sec:structure}

The structure should serve the purpose of your paper. However, there are certain conventions and based on those I suggest a structure that may be helpful.

\begin{enumerate}
\item Introduction: Explain what is the topic and why it is interesting for economists.
 \item  Related Literature Review: Briefly explain which other articles
  are relevant for the issue at hand and put the article in a broader context.
 \item  Model: Present the setup of the article.
 \item  Results: Present and discuss the results of the article. Avoid reproducing proofs, but make sure you understand them and give intuitions, illustrations, or summaries of the main arguments/mechanisms instead.
\item  Your Contribution: see above
  \item (Related Work: Briefly summarize related articles which appeared after the one you have been assigned. This can be merged with the ``related literature review'' as well.)
  \item Conclusion: Summarize your work and its message; possibly point out open
  questions.
\end{enumerate}

\section{Literature}
\label{sec:literature}
This sections covers various topics related to literature. The economics library offers courses for students on these topics that go much further into detail.

\subsection{Literature Search}
\label{sec:literature-search}
Starting from the assigned article the literature should be explored into two directions: Literature prior to the article and literature published after the article. Usually the article will contain a literature review that summarizes the related prior literature. To find related literature published later, it is possible to search for articles citing the assigned article (for example Google Scholar has this function). Also identifying keywords and searching for those -- in jstor.org or Google Scholar -- may yield related articles.

Several heuristics can be used to establish which are the most important articles. First, important articles tend to be cited more often (however note that relatively recent papers are naturally cited less). Second, the most important articles are discussed in detail in the literature sections of other articles. Third, the most important articles tend to be published in the most prestigious journals. A very incomplete and very subjective list of prestigious journals in the field of microeconomic theory and its applications is as follows:

\begin{itemize}
\item top five general interest journals: American Economic Review, Econometrica, Review of Economic Studies, Journal of Political Economy, Quarterly Journals of Economics
\item other general interest journals: Economic Journal, International Economic Review, Journal of the European Economic Association
\item top field journals: Journal of Economic Theory, Theoretical Economics, American Economic Journal: Microeconomics, RAND Journal of Economics, Games and Economic Behavior
\end{itemize}
Each of the three heuristics is imperfect, e.g. depending on the specifics of the  article many more field journals may be relevant. However, using common sense in combination with these heuristics will usually yield a short list of most important related papers.

\subsection{Reading}
\label{sec:reading}

I want to touch only briefly upon the topic of how to read an academic article for two reasons. First, a lot has been written on this topic and a quick web search will yield many answers. Second, I believe that there is no universal answer as reading styles differ from person to person.

The most important point is that the way one reads should depend on the purpose of reading. Consequently, academic articles should be read differently from poems, fiction or newspaper articles. The usual advice is to read title and abstract first. Then, to skim through the paper, checking the section titles, main results and possibly graphs and tables. The next step is to read introduction and conclusion. Finally, one can turn to the meat of the article (model, results, discussion). Eventually, one may check proofs, appendices, etc. The really important bit is that one may stop the process at every step. For example, one may realize after reading the abstract that the article is not really on the topic one is interested in and stop. Even if the article is related, reading of all parts may be unnecessary. Finally, it is useful to take notes and maybe write down summaries of those papers that are related to one's topic.

\subsection{Citations}
\label{sec:citations}

In terms of citation style, economists tend to use ``Author (year)''. However, more important than the actual style is that it is used consistently throughout the paper. The use of a citation tool is highly recommended, see the section on \LaTeX.

Students often ask for the right number of references. There is no general answer to this question as this highly depends on topic, structure and own contribution. Name dropping is not useful. That is, an article should only be cited if something about this article or a result/idea from the article is explicitly stated in the paper. The most general statement I dare to make is that three references are very likely to be too few and 23 are very likely to be too many for a seminar paper.

Finally, let me state the obvious: Direct quotations have to be clearly visible as such. Paraphrased sentences or paragraphs from other papers require a clear citation as well.

\section{Presentation}
\label{sec:presentation}

This section deals with various aspects of presentation and formatting.

\subsection{Academic Language}
\label{sec:academic-language}

A seminar paper is written in a formal language. Value judgments are avoided. Most importantly, the meaning has to be clear. In this context, formal language should not be understood as complicated. Short sentences are fine and ornamental elements should generally be avoided.
There are numerous style guides for academic writing available in book form as well as online.

\subsection{Layout and \LaTeX}
\label{sec:layout}

Using \LaTeX is highly recommended. (Almost all research papers in economics, mathematics, natural sciences and engineering are written in \LaTeX. \LaTeX { }is particularly useful for longer documents with mathematical content, e.g. a Master thesis.) One big advantage is the use of BibTeX as a system for citations which ensures consistent citations throughout the paper. The chair provides a template and links to online resources online, see \url{https://schottmueller.github.io/index.html/#Teaching}. 

One-half line spacing is recommended in \LaTeX (double spacing in Word). Font size 11 or 12 and margins of at least 2.5cm are recommended.

\subsection{Computation and Graphs}
\label{sec:computation-graphs}

\LaTeX{ }has the TIKZ package to generate professionally looking graphs. Jacques Cremer has written a mini-introduction to TIKZ, see \url{http://cremeronline.com/LaTeX/minimaltikz.pdf}, that is more than sufficient for the purposes of economists.

The chair provides some jupyter notebooks that explain how you can make professionally looking plots, numerically solve maximization problems and numerically solve (systems of) equations. The backend for all this is the Julia programming language, see \url{https://julialang.org}, but no prior knowledge of programming is required. The notebooks can be found under \url{https://github.com/schottmueller/juliaForMicroTheory}. To learn programming in julia from scratch, students may want to check out the tutorial on  \url{https://benlauwens.github.io/ThinkJulia.jl/latest/book.html}.\footnote{For, more examples where Julia is used to solve problems in (macro)economics, see \url{https://lectures.quantecon.org/jl/}.}


\section{Conclusion}
\label{sec:conclusion}

I want to conclude with two considerations. First, the role of supervision in the process of writing a seminar paper. It is recommended to have a midterm review with the supervisor. At this point, the student needs to have a detailed proposal for the own contribution and an outline for the seminar paper. The student should also have concluded his literature search and identified the most related articles. The meeting with the supervisor will mainly clarify whether the proposed own contribution is feasible and interesting. If the student is sufficiently confident, such a meeting can be skipped, i.e. the meeting is not mandatory.  

Second, I want to discuss to which extent a seminar paper is different from a Master thesis or a chapter in a PhD thesis or a published research article. Essentially, these are all forms of research papers and they differ only in emphasis and length. In a seminar paper, the emphasis is more on the literature (relative to the other forms of research papers). Roughly speaking, literature and own contributions have almost equal weight in a seminar paper. A (literature based) Master thesis puts slightly more weight on the own contribution and tends to be longer as students have more time. A chapter in a PhD thesis and a published research article will put substantially more weight on the own contribution and the weight on summarizing the literature will be much lower. \emph{Weight} and \emph{emphasis} have a double meaning in this context: on the one hand, weight relates to the quantity of space and on the other hand it relates to which extent the literature part determines the quality of the paper.


\bibliographystyle{chicago}
\bibliography{~/stuff/bibliography/references.bib}


\end{document}
